\documentclass[a4paper]{article}

%% Standard packages
\usepackage{times}
\usepackage{a4}
\usepackage{graphicx}
\usepackage{mathtools}
\usepackage{amsfonts}
\usepackage{amssymb}
\usepackage{hyperref}
\usepackage{color}

%% Unicode support
\usepackage[utf8x]{inputenc}
\usepackage{ucs}
\usepackage{autofe}


%% Agda
\usepackage{agda/agda}
\usepackage{catchfilebetweentags}


%% PDF metainformation
\usepackage{datetime}
\usepackage{ifpdf}
\ifpdf
\pdfinfo{
    /Author (Joao Paulo Pizani Flor, Wout Elsinghorst)
    /Title (Proving Compiler Correctness with Dependent Types)
    /Keywords (Agda, Dependently-typed programming, Compiler correctness, typed bytecode)
    /CreationDate (D:\pdfdate)
}
\fi


%% LaTeX meta-information
\title{Proving Compiler Correctness with Dependent Types}

\date{\today}
\author {
    João Paulo Pizani Flor \texttt{<j.p.pizaniflor@students.uu.nl>} \\
    \and Wout Elsinghorst \texttt{<w.l.elsinghorst@students.uu.nl>} \\
}



%% The document itself
\begin{document}
    \maketitle

    \section{Introduction}
    \label{sec:intro}
        In this report we describe our work for the final project of the master course
        "Theory of Programming and Types" at Utrecht University. Our research involves
        the question of compiler correctness, i.e, giving a \emph{specification} for
        a language (a semantics), and proving that a compiler for that language respects
        the given semantics.

        Concretely, our notion of correctness depends on two semantics, respectively for the source
        and object languages of the compiler. We define a denotational semantics (eval) for the source
        language, as well as an operational semantics (exec) for the target machine. Correctness
        states that evaluation is equal to compilation composed with execution.

        More specifically, we were interested in having \emph{machine-checked} proofs of correctness,
        i.e, proofs written in the language of an interactive proof assistant. Some initiatives in
        this direction are already being taken, most famously CompCert, a formally verified compiler
        for the C language, which had its proof written in the \emph{Coq} proof assistant.

        Our approach, besides using the \emph{Agda} programming language instead of emph{Coq},
        also differs from CompCert in that it uses dependent types as a way to make the
        source and object languages (as well as the compiler) have some meta-theoretical properties
        \emph{by construction}, instead of proving them separately.

        Dependent types allow us to embed, as indices to the language definitions and in the type
        of the ``compile'' function, constraints which make these definitions correct. %%%

        The rest of the report is organized as follows: on section \ref{sec:goals} we present the
        related papers which served as the basis for our research, and state precisely which were
        our contributions. On section \ref{sec:basic} we define the notion of correctness that we
        use, and present the definitions for our \emph{source} and \emph{object} languages,
        the compiler itself and the proof of correctness for that compiler. Section \emph{sec:lifiting}
        introduces a smarter version of the object language, in which sharing is captured, and a
        compiler which produces shared code. We then proceed to describe how,
        given a correctness proof for a ``naïve'' compiler, we can derive the correctness proof of 
        the smarter version.

    \section{Related work/Goals}
    \label{sec:goals}
        % Mention two papers we based our work on
        On researching the topic of compiler correctness in the context of dependent types, we first encountered
        the paper ``A type-correct, stack-safe, provably correct expression compiler in Epigram'' \cite{typed-stack-safe-compiler}.
        In this paper, a \emph{very simple} (but \emph{typed}) expression language is presented,
        along with a \emph{typed bytecode} definition. The correctness of this compiler is then formulated and proven.
        
        Both language definitions, the semantics for each of them, the compiler and its correctness proof
        were all written in Epigram, a dependently-typed programming language. Therefore,
        even though the authors admit that the paper carries no real novelty value related to compiler
        construction, we still felt this paper could serve well as the ``basis'' for our project, as they
        treat compilation of \emph{typed source} to \emph{typed bytecode}, and also use a dependently-typed
        implementation language.

        So our work started with trying to extend the source language from \cite{typed-stack-safe-compiler},
        and add constructs to it that made it more powerful. Then we came accross the paper
        ``Proving Correctness of Compilers using Structured Graphs''\cite{compiler-correctness-structured-graphs}.
        In this paper, the author discusses the issue of \emph{sharing} of bytecode, that is, that
        some high-level language constructs (typically control flow related constructs,
        such as exception handling of conditional branching) map into low-level code which has
        \emph{shared blocks of code among different execution paths}.

        One way to compile these control flow constructs would be to extend the bytecode language with
        explicit jumps and labels, a solution which is often taken in ``real-world'' compilers,
        but which makes analyzing bytecode and proving compiler correctness much more complex.
        A simpler way to handle these situations is to just \emph{replicate} the shared code when
        compiling constructs which generate different execution paths.

        %% This is easier, but bla bla

        % Our wish to join the contributions of both papers
        % What adaptations we did to each approach to help them work together:
            % Extension in the Src to reveal "sharing opportunities"
            % Make functors and proofs of the lifting approach indexed

    \section{Basic correctness}
    \label{sec:basic}
        % Introduce compiler correctness in general (no models mentioned)
        % Talk about our model with sequencing and the "vector values"
        % Code for Src and Bytecode, and correctness proof

    \section{Lifiting to shared setting}
    \label{sec:lifiting}
        % What is the sharing opportunity (diagram, just Src code example)
        % A different representation of Bytecode (graph) that could capture sharing
            % Give example of a "BytecodeG" term...
        % If we make Bytecode into functor, we can have different representations by
        % applying fixed-point-like operators
          % Htree, HGraph (iso, not iso really)

        % State the "lifting" theorems
            % Grand finale: correctG
           

    \section{Conclusions}
    \label{sec:conclusions}
        % Summarize contributions (again?)

        % Points where we got stuck (holes, postulates)
            % Some possibilities of attempts, if we had more time

        % Difficulties encountered along the way
            % Ideas of why they happened
                % Tool-related (Agda)
                % Domain-related (problem inherently hard)



    %% \bibliographystyle{plain}
    %% \bibliography{../references.bib}

\end{document}
