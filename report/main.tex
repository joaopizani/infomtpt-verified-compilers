\documentclass[a4paper]{article}

%% Standard packages
\usepackage{times}
\usepackage{a4}
\usepackage{graphicx}
\usepackage{mathtools}
\usepackage{amsfonts}
\usepackage{amssymb}
\usepackage{hyperref}
\usepackage{color}

%% Unicode support
\usepackage[utf8x]{inputenc}
\usepackage{ucs}
\usepackage{autofe}


%% Agda
\usepackage{agda/agda}
\usepackage{catchfilebetweentags}


%% PDF metainformation
\usepackage{datetime}
\usepackage{ifpdf}
\ifpdf
\pdfinfo{
    /Author (Joao Paulo Pizani Flor, Wout Elsinghorst)
    /Title (Proving Compiler Correctness with Dependent Types)
    /Keywords (Agda, Dependently-typed programming, Compiler correctness, typed bytecode)
    /CreationDate (D:\pdfdate)
}
\fi


%% LaTeX meta-information
\title{Proving Compiler Correctness with Dependent Types}

\date{\today}
\author {
    João Paulo Pizani Flor \texttt{<j.p.pizaniflor@students.uu.nl>} \\
    \and Wout Elsinghorst \texttt{<w.l.elsinghorst@students.uu.nl>} \\
}



%% The document itself
\begin{document}
    \maketitle

    \section{Introduction}
    \label{sec:intro}
        % General motivations, etc.
        % Formalize correctness in terms of semantics
        % How dependent types help.

    \section{Goals/Contributions}
    \label{sec:goals}
        % Mention two papers we based our work on
        % Our wish to join the contributions of both papers
        % What adaptations we did to each approach to help them work together:
            % Extension in the Src to reveal "sharing opportunities"
            % Make functors and proofs of the lifting approach indexed

    \section{Basic correctness}
    \label{sec:basic}
        % Introduce compiler correctness in general (no models mentioned)
        % Talk about our model with sequencing and the "vector values"
        % Code for Src and Bytecode, and correctness proof

    \section{Lifiting to shared setting}
    \label{sec:lifiting}
        % What is the sharing opportunity (diagram, just Src code example)
        % A different representation of Bytecode (graph) that could capture sharing
            % Give example of a "BytecodeG" term...
        % If we make Bytecode into functor, we cant have different 

    \section{Conclusions}
    \label{sec:conclusions}
        % Summarize contributions (again?)
        % Points where we got stuck (holes, postulates)
            % Some possibilities of attempts, if we had more time

        % Difficulties encountered along the way
            % Ideas of why they happened
                % Tool-related (Agda)
                % Domain-related (problem inherently hard)



    %% \bibliographystyle{plain}
    %% \bibliography{../references.bib}

\end{document}
