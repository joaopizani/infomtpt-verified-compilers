\section{Problem description}

    \subsection{Context}
\begin{frame}[fragile]
            \frametitle{Source code, Stack code, eval, compile, exec}

            \begin{itemize}
                \item The type of the interpreter states type preservation
            \end{itemize}

            \ExecuteMetaData[code-tex/Code.tex]{nat}
\end{frame}


    \subsection{Compiler correctness}
        \begin{frame}
            \frametitle{What does "correct" mean?}

            \begin{itemize}
                \item Commutative diagram?
            \end{itemize}
        \end{frame}


    \subsection{Extending the source language}
        \begin{frame}
            \frametitle{Extension}

            \begin{itemize}
                \item Sequencing expressions
                \item Enables "branching", but with shared suffix
            \end{itemize}
        \end{frame}

        \begin{frame}
            \frametitle{Code duplication}

            \begin{itemize}
                \item The "normal" compile function will duplicate the common suffix
                \item Having Bytecode defined as graph (structured graph) instead of tree
                    would solve this problem
                    \begin{itemize}
                        \item But proofs would be harder
                    \end{itemize}
            \end{itemize}
        \end{frame}

        \begin{frame}
            \frametitle{Graph and tree representations are equivalent}

            \begin{itemize}
                \item Equations
                \item Make Bytecode a functor (BytecodeF)
                    \begin{itemize}
                        \item Two type-level fixpoint operators give two different Bytecode versions:
                        \begin{itemize}
                            \item HTree gives us BytecodeT
                            \item HGraph gives us BytecodeG
                        \end{itemize}
                    \end{itemize}

                \item Now for the implementation
            \end{itemize}
        \end{frame}


